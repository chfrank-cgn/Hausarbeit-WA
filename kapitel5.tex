%
%	Fazit
%

\pagebreak
\section{Summary}

\onehalfspacing

From our analysis, we can see that it makes a lot of sense to focus the the Kölle for Future blog posts on the two major crises of our time, the Climate Emergency and the COVID-19 pandemic.

Using wp-statistics for web analytics was initially an easy option; however, if we want to continue the in-depth analysis of the performance of our blog, we might consider moving to another tool or platform.

The overall result is not a surprise; these two issues are the most talked-about issues on all media and are on everybody's mind all the time. The data showed us some more actionable items, such as putting more focus on mobile users and asking us to make sure that we cater to browsers and platforms that are not mainstream.

But, most importantly, the data showed us that we were able to engage with the KFF community and the broader network of climate activists with the help of our blog. When it's de rigueur to social distance, it is vital to use alternative means to reach out, especially means that do not require physical contact. 

From the past year, we can conclude that digital activism works, both as a support for the community and as a way of protest. Quite surprisingly, we found out that our politicians indeed pay attention to social media.

Maintaining connections during the pandemic is the most crucial part of combating loneliness and building up resilience, and it is the focus of our outreach work.

Happy Activism!
