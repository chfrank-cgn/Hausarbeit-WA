%
%	Fazit
%

\pagebreak
\section{Summary}

\onehalfspacing

From the analysis, we can conclude that it does make sense to focus the posts on my blog on the two major crises of our time, the Climate Emergency and the COVID-19 pandemic.

The result is not entirely surprising, as these are the most talked-about issues on all media and are on everybody's mind all the time. However, having the assumptions being proved by data science through a thorough data analysis helps a lot and will guide me in further postings and the blog's development.

The power is in the data - even for a small blog (and thus a small data set), analyzing the data is a worthwhile thing to do and will lead to exciting and actionable results.

As we saw, Plausible is a unique open-source tool for web analytics. There are many other open-source tools available to support data science and analysis; there is also a big community around these tools and the subject of Big Data and Data Science. 

In this paper, I was merely able to scratch the surface, but I hope I could provide you with at least some valuable insights and pointers to start with; all available raw data is in the two data notebooks and the sidebar of my blog.

Happy Analysis!
