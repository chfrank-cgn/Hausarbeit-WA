%
%	Fazit
%

\pagebreak
\section{Summary}

\onehalfspacing

From the analysis, we can see that it makes a lot of sense to focus the posts on the Kölle for Future blog on the two major crises of our time, the Climate Emergency and the COVID-19 pandemic.

Using wp-statistics for web analytics was initially an easy option, however, in case we want to continue in-depth analysis of the performance of our blog, we might consider moving to another tool or platform.

The overall result is not really a surprise, these two issues are the most talked-about issues on all media and are on everybody's mind all the time. The data showed us also some more actionable items, such as putting more focus on mobile users and is also asking us to make sure that we cater for browsers and platforms which are not mainstream.

But, most importantly, the data showed us that we can engage with the KFF community and the wider network of climate activists with the help of our blog. In times where it's de rigueur to socially distance, it is very important to use alternative means to reach out, especially means that do not require physical contact. 

From the past year we can conclude that digital activism works, both as a support for the community and also as way of protest - quite surprisingly we found out that our politicians indeed pay attention to social media.

Maintaining connections during the pandemic is the most important part of combating loneliness and building up resilience, and it is the focus of our outreach work.

Happy Activism!
