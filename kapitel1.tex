%
%	Einfuehrung
%

\pagebreak
\section{Web Traffic Analysis}

\onehalfspacing

\subsection{Digital Activism}

In this paper, I will look at my blog's traffic data, analyze that data, and then lay the foundation to predict a blog post's performance based on its content.

Web analytics is the domain of the big search engines, and Google Analytics and Google AdSense are the market leaders, followed by Microsoft Bing Web Analytics. In all of Social Media and Social Media Marketing, web analytics plays a crucial role in evaluating a website's performance and forms the basis of automated advertising placement.

Advertising, as much as we might dislike it, pays for most of the content we consume.

Page views, bounce rate, and unique visitors are key metrics to evaluate a website and the currency that fuels the internet. Every marketeer or web site owner will use these metrics to analyze performance and identify areas for growth; many tools for analysis have become available in the last couple of years, some of them open-source, some closed-source.

Generally speaking, more traffic can potentially lead to more business opportunities. It is mandatory for a commercial website to monitor its web analytics data daily and act immediately on any anomaly.

However, a blog does not necessarily have commercial interests and might be an outlet for personal interests or interactions. Why would we want to look at web analytics anyway?

\subsection{Gender-neutral Pronouns}

As we move towards a more inclusive and gender-fluid society, it's time to rethink the usage of gendered pronouns in scientific texts. Two well-known professors from UCLA, Abigail C. Saguy and Juliet A. Williams, argue that it makes a lot of sense to use singular they/them instead: "The universal singular they is inclusive of people who identify as male, female or nonbinary."\footnote{\textit{Saguy, A. (2020)}: Why We Should All Use They/Them Pronouns. \cite{pronouns}} Throughout this paper, I'll attempt to follow their suggestion and invite my readers to do the same in future articles, and support an inclusive approach through gender-neutral language. Thank you!

