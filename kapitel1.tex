%
%	Einfuehrung
%

\pagebreak
\section{Introduction}

\onehalfspacing

\subsection{Digital Activism}

Since early 2020, the world is in various stages of lockdown, with a mandate for social distancing. We've suspended meetings with family, friends, and colleagues. We've canceled all large gatherings.

This has a profound effect on all activism, including climate activism. We had to move most activities to the digital realm. Organizing protests and keeping the momentum alive is not easy during lockdown. Activists will have to perform online and offline actions to keep the pressure up and maintain awareness of the impending climate emergency.\footnote{See \textit{Vinter, R. (2021)}: Climate protesters gather in person and online. \cite{climateProtest}} Actions could include Twitter storms and massive Zoom calls, in addition to the more traditionals rallies.\footnote{See \textit{Morresi, E. (2021)}: Protest in a pandemic. \cite{pandemicProtest}} 

But not all is doom and gloom - moving part of the activism into the aether did enable a much wider reach for some groups and actions. As participation is no longer tied to location, new world-wide groups and activist collectives emerged. 

One example is \href{https://fffdigital.carrd.co/}{FFF Digital} with their motto "We are still here". Having only digital means for direct action can be tiring and cumbersome, but FFF Digital gives a voice to the Global South and most affected people and areas (\href{https://fridaysforfuture.org/country/mapa/}{MAPA}); a voice that can be heard all over the globe. \href{https://twitter.com/fffmapa}{FFF MAPA} is now a distinct Twitter channel amongst all other countries.

There are many forms of digital activism, mainly of social media. Here we will focus on a more traditional medium, a climate activists' blog for our home town, Cologne.

\subsection{Gender-neutral Pronouns}

Our society is becoming more open, inclusive and gender-fluid, and now I think it's time to think about using gender-neutral pronouns in scientific texts, too. Two well-known researches, Abigail C. Saguy and Juliet A. Williams, both from UCLA, propose to use singular they/them instead: "The universal singular they is inclusive of people who identify as male, female or nonbinary."\footnote{\textit{Saguy, A. (2020)}: Why We Should All Use They/Them Pronouns. \cite{pronouns}} The aim is support an inclusive approach in science through the use of gender-neutral language. 

In this paper I'll attempt to follow this suggestion and invite all my readers to do the same for future articles. Thank you!

If you're not sure about the definitions of gender and sex, and how to use them, have a look at the definitions\footnote{See \textit{APA (2021)}: Definitions Related to Sexual Orientation. \cite{apaDefinitions}} by the American Psychological Association.

