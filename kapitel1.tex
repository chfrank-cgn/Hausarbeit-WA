%
%	Einfuehrung
%

\pagebreak
\section{Introduction}

\onehalfspacing

\subsection{Digital Activism}

In times of Covid-19 climate activisim has partially moved to the digital realm.

Guardian\footnote{\textit{Morresi, E. (2021)}: Protest in a pandemic. \cite{pandemicProtest}} and Guardian\footnote{\textit{Vinter, R. (2021)}: Climate protesters gather in person and online. \cite{climateProtest}} 

Entirely digital \href{https://fffdigital.carrd.co/}{FFF Digital}

\subsection{Gender-neutral Pronouns}

As we move towards a more inclusive and gender-fluid society, it's time to rethink the usage of gendered pronouns in scientific texts. Two well-known professors from UCLA, Abigail C. Saguy and Juliet A. Williams, argue that it makes a lot of sense to use singular they/them instead: "The universal singular they is inclusive of people who identify as male, female or nonbinary."\footnote{\textit{Saguy, A. (2020)}: Why We Should All Use They/Them Pronouns. \cite{pronouns}} Throughout this paper, I'll attempt to follow their suggestion and invite my readers to do the same in future articles, and support an inclusive approach through gender-neutral language. Thank you!
A handy reference to the terms\footnote{\textit{APA (2021)}: Definitions Related to Sexual Orientation. \cite{apaDefinitions}}

